\let\negmedspace\undefined
\let\negthickspace\undefined
\documentclass[journal,12pt,twocolumn]{IEEEtran}
\usepackage{gensymb}
\usepackage{polynom}
\usepackage{amssymb}
\usepackage[cmex10]{amsmath}
\usepackage{amsthm}
\usepackage{stfloats}
\usepackage{bm}
\usepackage[misc]{ifsym}

\usepackage{longtable}
\usepackage{enumitem}
\usepackage{mathtools}
\usepackage[applemac]{inputenc}
\usepackage{tikz}
\usepackage[breaklinks=true]{hyperref}
\usepackage{listings}
    \usepackage{color}                                            %%
    \usepackage{array}                                            %%
    \usepackage{longtable}                                        %%
    \usepackage{calc}                                             %%
    \usepackage{multirow}                                         %%
    \usepackage{hhline}                                           %%
    \usepackage{ifthen}                                           %%
  
\usepackage{lscape}     
\usepackage{tfrupee}
\usepackage{parskip}

\pagecolor{white}
\DeclareMathOperator*{\Res}{Res}
\DeclareMathOperator*{\equals}{=}

\hyphenation{op-tical net-works semi-conduc-tor}
\def\inputGnumericTable{}  
\graphicspath{{./images/}}


\begin{document}

	\title{Assignment 3 : Example 9}
		\author{ Abhay Shankar K : cs21btech11001}

		\maketitle

		\bigskip

	\providecommand{\brak}[1]{\ensuremath{\left(#1\right)}}
	\providecommand{\sbrak}[1]{\ensuremath{\left[#1\right]}}
	\providecommand{\abs}[1]{\left\vert#1\right\vert}
	\providecommand{\norm}[1]{\left\lVert#1\right\rVert}
	\newcommand{\solution}{\noindent \textbf{Solution: }}
	\newcommand{\question}{\noindent \textbf{Question: }}

	\newcommand{\myvec}[1]{\ensuremath{\begin{pmatrix}#1\end{pmatrix}}}
	\let\vec\mathbf


	\question
	
	
	Consider the frequency distribution table (given below), which gives the weights of $38$ students in a class.
	
	\begin{enumerate}[label = \brak{\textbf{\roman*}}]
	
		\item Find the probability that the weight of a student in a class lies in the interval $46 - 50$.
		
		\item Give two events in this context, one having probability $0$, and the other having probability $1$.
		
	\end{enumerate}
	
	
	\begin{table}[!htb]
	
		\centering
		
		\caption{Distribution of weights of students in a class}
		\input{tables/assig3}
		
		
		\label{table : given_table}
		
	\end{table}
	
	
	\solution
	
	\begin{enumerate}[label = \brak{\textbf{\roman*}}]
	\item
	
	Let us define a random variable X, which represents the weight class of a student. The range of X and the corresponding weight class is given in the table below.
	
	\begin{table}[h!tb]
	
		\centering
		
		\caption{Range of X mapped to weight classes from ~\ref{table : given_table}}
		\input{tables/rand_var}
		
		
		\label{table : range_table}
		
	\end{table}
	
	
	\begin{align}
		\therefore P_X \brak{4} = \frac{3}{38} = 0.079
	\end{align}
	
	
	The probability that the weight of a student is in the weight class \brak{46 - 50} is $\underline{0.079}.$ 
		
	
	\item
	
	\begin{itemize}
	
		\item The probability that the weight of a student is greater that 75 is $0$.
		
		\item The probability that the weight of a student is in the interval \sbrak{31, 75} is $1$.
	
	\end{itemize}
	
	\end{enumerate}
	
	\begin{figure}[!htb]
	
		\includegraphics[width = \columnwidth]{assig3_hist}
		\caption{Histogram of given data (produced using python)}
		
	\end{figure}
\end{document}
	
	
	
	
	
	
	
	
	
	
	
	
	
	
	
	
	
	
	
	
	
	
	
	
	
\end{document}