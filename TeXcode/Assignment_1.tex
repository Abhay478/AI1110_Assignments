\documentclass[11pt, oneside]{amsart}   	% use "amsart" instead of "article" for AMSLaTeX format
\usepackage{geometry} 
\usepackage{polynom}              		% See geometry.pdf to learn the layout options. There are lots.
\geometry{letterpaper}                   		% ... or a4paper or a5paper or ... 
%\geometry{landscape}                		% Activate for rotated page geometry
\usepackage[parfill]{parskip}    		% Activate to begin paragraphs with an empty line rather than an indent
\usepackage{graphicx}				% Use pdf, png, jpg, or eps§ with pdflatex; use eps in DVI mode
								% TeX will automatically convert eps --> pdf in pdflatex		
\usepackage{amssymb}
\usepackage{gensymb}
\usepackage{amsmath}
\usepackage{enumitem}
 \usepackage{mathtools}
 %SetFonts

%SetFonts


\title{Assignment 1 : Question 11 (a)}
\author{Abhay Shankar K : cs21btech11001}
%\date{}							% Activate to display a given date or no date

\begin{document}
\maketitle
%\section{}
%\subsection{}
Given:
\begin{itemize}
	\item $f(x) = x^3 + (kx + 8)x + k$ 
	\item Sum of remainders of $f(x)$ after dividing by $(x + 1)$ and $(x - 2)$ is $1$
\end{itemize}
Find:
\begin{itemize}
	\item Remainders of $f(x)$ after dividing by $(x + 1)$ and $(x - 2)$
	\item The value of $k$
\end{itemize}

\textbf{Solution :}

By the remainder theorem, \\
	\quad The remainder after dividing a polynomial $p(x)$ by $(x - r)$ is equal to $p(r)$.

	 Therefore,
	 
	\begin{math}
  		f(x)\ \%\ (x + 1) = f(-1) \\
  		f(x)\ \%\ (x - 2) = f(2) \\
				\\
 		f(-1) = (-1)^3 + (k(-1) + 8)*(-1) + k \\
		i.e. \quad \ = -1 + k - 8 + k \\
		i.e. \quad \ = \underline{2k - 9}  \\
		\\
		f(2) \ = 2^3 + (2k + 8)*2 + k \\
		i.e. \quad = 8 + 4k + 16 + k \\
		i.e. \quad  = \underline{5k + 24}
				  \\
	\end{math}
		  
	Given that $(2k - 9) + (5k + 24) = 1$
	\\	
	Rearranging, we get $7k = -14$ \\
		\\
	Therefore \underline{k = -2}\\
		\\
	Substituting the value of $k$, we get \\
	\begin{math}
		2k - 9 = -13 \\
		5k + 24 = 14 \\
		\\
\text{Therefore, the remainders are :}
\\
	f(x)\ \%\ (x + 1) = -13 \\
	f(x)\ \%\ (x - 2) = 14 \\
		\end{math}
\\
		
	 


\end{document}  