%%%%%%%%%%%%%%%%%%%%%%%%%%%%%%%%%%%%%%%%%%%%%%%%%%%%%%%%%%%%%%%
%
% Welcome to Overleaf --- just edit your LaTeX on the left,
% and we'll compile it for you on the right. If you open the
% 'Share' menu, you can invite other users to edit at the same
% time. See www.overleaf.com/learn for more info. Enjoy!
%
%%%%%%%%%%%%%%%%%%%%%%%%%%%%%%%%%%%%%%%%%%%%%%%%%%%%%%%%%%%%%%%


% Inbuilt themes in beamer
\documentclass{beamer}

% Theme choice:
\usetheme{CambridgeUS}

% Title page details: 

\usepackage{polynom}
\usepackage{amssymb}
\usepackage{amsmath}

\usepackage{bm}
\usepackage[misc]{ifsym}
%\setlength{\columnseprule}{1pt}
%\def\columnseprulecolor{\color{blue}}

\usepackage{longtable}
\usepackage{enumitem}
\usepackage{mathtools}
\usepackage[applemac]{inputenc}
\usepackage{tikz}
\usepackage{hyperref}
\usepackage{listings}
    \usepackage{color}                                            %%
    \usepackage{array}                                            %%
    \usepackage{longtable}                                        %%
    \usepackage{calc}                                             %%
    \usepackage{multirow}                                         %%
    \usepackage{hhline}                                           %%
    \usepackage{ifthen}                                           %%
  
\usepackage{lscape}     
\usepackage{tfrupee}
\usepackage{parskip}

\pagecolor{white}
\DeclareMathOperator*{\Res}{Res}
\DeclareMathOperator*{\equals}{=}
\DeclareMathOperator*{\pipe}{|}

\hyphenation{op-tical net-works semi-conduc-tor}
\def\inputGnumericTable{}  
\graphicspath{{./images/}}


\begin{document}
	\title{Assignment 7}
		\author{ Abhay Shankar K : CS21BTECH11001}

	\begin{frame}
    		\titlepage 
	\end{frame}


	\providecommand{\brak}[1]{\ensuremath{\left(#1\right)}}
	\providecommand{\sbrak}[1]{\ensuremath{\left[#1\right]}}
	\providecommand{\cbrak}[1]{\ensuremath{\left\{#1\right\}}}
	\newcommand{\req}{\noindent \textbf{Required: }}
	\providecommand{\rpr}[2]{\ensuremath{P_{#1}\left(#2\right)}} %random variable notation
	\providecommand{\spr}[1]{\ensuremath{P\left(#1\right)}} %simple notation
	\providecommand{\cpr}[2]{\ensuremath{\spr{#1 \pipe #2}}} %conditional probability


	\begin{frame}{Outline}
    		\tableofcontents
	\end{frame}


	\section{Question}
		\begin{frame}{Question}

			\begin{block}{Exercise 13.3 : Question 2}
			
				A bag contains $4$ red and $4$ black balls, and another bag contains $2$ red and $6$ black balls. One of the bags is chosen at random and a ball is drawn from it which is found to be red. Find the probability that the ball is drawn from the first bag.
			
			\end{block}

		\end{frame}


	\section{Random Variable definition}
		\begin{frame}{Definitions}

			\begin{block}{X}
			
				Let the random variable X represent the bag chosen.
				\begin{align}
					X &= 0 \leftarrow \text{First bag} \nonumber \\
					X &= 1 \leftarrow \text{Second bag}
						\label{Xrange}
				\end{align}

			\end{block}

			\begin{block}{Y}
			
				Let the random variable Y represent the colour of the ball chosen.
				\begin{align}
					Y &= 0 \leftarrow \text{Red ball} \nonumber \\
					Y &= 1 \leftarrow \text{Black ball}
						\label{Yrange}
				\end{align}

			\end{block}

			\begin{block}{Reframing}
			
				\req \raggedright \cpr{\text{First Bag}}{\text{Red ball}} = \cpr{X = 0}{Y = 0}

			\end{block}

		\end{frame}

	\section{Bayes' theorem}
	
		\begin{frame}{Bayes' theorem}
		
			The problem is a classic application of Bayes' theorem.
			
			\begin{block}{}
			
				Bayes' theorem states :
				
				\begin{align}
					\frac{\cpr{A}{B}}{\cpr{B}{A}} = \frac{\spr{A}}{\spr{B}} 
						\label{Bayes_thm}
				\end{align}
				
				where A and B are any two events. 
			\end{block}

		\end{frame}

	\section{Table}
		\begin{frame}{Probabilities}
	
			\begin{table}[h!b]
	
				\centering
				\input{tables/assig7_table}
		
				\label{table}
		
			\end{table}
	
			Using the formula
			\begin{align}
				\rpr{Y}{0} &= \rpr{X}{0} \times \cpr{Y = 0}{X = 0} \nonumber \\
				&+ \rpr{X}{1} \times \cpr{Y = 0}{X = 1}
					\label{prob_form}
			\end{align}

		\end{frame}

	\section{Calculation}
		\begin{frame}{Calculation}
			Substituting the operational variables from the table  into equation ~\eqref{Bayes_thm} and subsequently rearranging,
	
	
			\begin{align}
				\frac{\cpr{X = 0}{Y = 0}}{\cpr{Y = 0}{X = 0}} &= \frac{\rpr{X}{0}}{\rpr{Y}{0}} \\
					\nonumber \\
				\implies \frac{\cpr{X = 0}{Y = 0}}{0.5} &= \frac{0.5}{0.375} \\
					\nonumber \\
				\implies \cpr{X = 0}{Y = 0} &= \frac{2}{3}
					\label{soln}
			\end{align}
	
	
			Therefore, the probability that the ball is drawn from the first bag is $\underline{\frac{2}{3}}$ .
		\end{frame}

\end{document}