\let\negmedspace\undefined
\let\negthickspace\undefined
\documentclass[journal,12pt,twocolumn]{IEEEtran}
\usepackage{gensymb}
\usepackage{polynom}
\usepackage{amssymb}
\usepackage[cmex10]{amsmath}
\usepackage{amsthm}
\usepackage{stfloats}
\usepackage{bm}
\usepackage[misc]{ifsym}
%\setlength{\columnseprule}{1pt}
%\def\columnseprulecolor{\color{blue}}

\usepackage{longtable}
\usepackage{enumitem}
\usepackage{mathtools}
\usepackage[applemac]{inputenc}
\usepackage{tikz}
\usepackage[breaklinks=true]{hyperref}
\usepackage{listings}
    \usepackage{color}                                            %%
    \usepackage{array}                                            %%
    \usepackage{longtable}                                        %%
    \usepackage{calc}                                             %%
    \usepackage{multirow}                                         %%
    \usepackage{hhline}                                           %%
    \usepackage{ifthen}                                           %%
  
\usepackage{lscape}     
\usepackage{tfrupee}
\usepackage{parskip}

\pagecolor{white}
\DeclareMathOperator*{\Res}{Res}
\DeclareMathOperator*{\equals}{=}

\hyphenation{op-tical net-works semi-conduc-tor}
\def\inputGnumericTable{}  
\graphicspath{{./images/}}

\begin{document}

	\title{Assignment 6 : Exercise 13.3 Question 2}
		\author{ Abhay Shankar K : cs21btech11001}

		\maketitle

		\bigskip

	\providecommand{\brak}[1]{\ensuremath{\left(#1\right)}}
	\providecommand{\sbrak}[1]{\ensuremath{\left[#1\right]}}
	\providecommand{\cbrak}[1]{\ensuremath{\left\{#1\right\}}}
	\newcommand{\solution}{\noindent \textbf{Solution: }}
	\newcommand{\question}{\noindent \textbf{Question: }}
	\newcommand{\req}{\noindent \textbf{Required: }}
	\providecommand{\rpr}[2]{\ensuremath{P_{#1}\left(#2\right)}} %random variable notation
	\providecommand{\spr}[1]{\ensuremath{P\left(#1\right)}} %simple notation
	\providecommand{\cpr}[2]{\ensuremath{\spr{#1\ |\ #2}}} %conditional probability
	
	
	\question
	
	
	A bag contains $4$ red and $4$ black balls, and another bag contains $2$ red and $6$ black balls. One of the bags is chosen at random and a ball is drawn from it which is found to be red. Find the probability that the ball is drawn from the first bag.
	
	
	\solution
	
	
	Let the random variable X represent the bag chosen.
	\begin{align}
		X &= 0 \leftarrow \text{First bag} \nonumber \\
		X &= 1 \leftarrow \text{Second bag}
			\label{Xrange}
	\end{align}
	
	
	Let the random variable Y represent the colour of the ball chosen.
	\begin{align}
		Y &= 0 \leftarrow \text{Red ball} \nonumber \\
		Y &= 1 \leftarrow \text{Black ball}
			\label{Yrange}
	\end{align}


	\req 
	
	\raggedright \cpr{\text{First Bag}}{\text{Red ball}} = \cpr{X = 0}{Y = 0}
	
	
	Bayes' theorem states :
	\begin{align}
		\frac{\cpr{A}{B}}{\cpr{B}{A}} = \frac{\spr{A}}{\spr{B}} 
			\label{Bayes_thm}
	\end{align}
	
	
	The various probabilities required for the computation are tabularized below :
	
	
	\begin{table}[h!b]
	
		\caption{Probabilities}
		\input{tables/assig7_table}
		
		\label{table}
		
	\end{table}
		
	
	Using the formula
	\begin{align}
		\rpr{Y}{0} &= \rpr{X}{0} \times \cpr{Y = 0}{X = 0} \nonumber \\
		&+ \rpr{X}{1} \times \cpr{Y = 0}{X = 1}
			\label{prob_form}
	\end{align}
	
	
	\newpage


	Substituting the operational variables from table ~\ref{table} into equation ~\eqref{Bayes_thm} and subsequently rearranging,
	
	
	\begin{align}
		\frac{\cpr{X = 0}{Y = 0}}{\cpr{Y = 0}{X = 0}} &= \frac{\rpr{X}{0}}{\rpr{Y}{0}} \\
			\nonumber \\
		\implies \frac{\cpr{X = 0}{Y = 0}}{0.5} &= \frac{0.5}{0.375} \\
			\nonumber \\
		\implies \cpr{X = 0}{Y = 0} &= \frac{2}{3}
			\label{soln}
	\end{align}
	
	
	Therefore, the probability that the ball is drawn from the first bag is $\underline{\frac{2}{3}}$ .
				
				
\end{document}
	
	

	
	
	
	