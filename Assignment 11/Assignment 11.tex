
\documentclass{beamer}

% Theme choice:
\usetheme{CambridgeUS}

% Title page details: 

\usepackage{polynom}
\usepackage{amssymb}
\usepackage{amsmath}

\usepackage{bm}
\usepackage[misc]{ifsym}

\usepackage{enumitem}
\usepackage{mathtools}
\usepackage[applemac]{inputenc}
\usepackage{tikz}

\usepackage{parskip}

\DeclareMathOperator*{\Res}{Res}
\DeclareMathOperator*{\equals}{=}
\DeclareMathOperator*{\pipe}{|}


\hyphenation{op-tical net-works semi-conduc-tor}
\def\inputGnumericTable{}  
\graphicspath{{./images/}}


\begin{document}
\newcommand{\bfr}[2]{\section{#1} \begin{frame}{#1} #2 \end{frame}}

	\title{Assignment 11}
		\author{ Abhay Shankar K: CS21BTECH11001}
\date{}
	\begin{frame}
    		\titlepage	
	\end{frame}

	\begin{frame}{Outline}
    		\tableofcontents
	\end{frame}

	\providecommand{\brak}[1]{\ensuremath{\left(#1\right)}}
	\providecommand{\rpr}[2]{\ensuremath{P_{#1}\left(#2\right)}} %random variable notation
	\providecommand{\spr}[1]{\ensuremath{P\left(#1\right)}} %simple notation
	\newcommand{\abs}[1]{\left| #1 \right|}
	\providecommand{\sbrak}[1]{\ensuremath{\left[#1\right]}}
    \newcommand{\myvec}[1]{\ensuremath{\begin{pmatrix}#1\end{pmatrix}}}
	
	\providecommand{\pmf}[2]{\ensuremath{f_{#2}\left(#1\right)}}
	\providecommand{\cdf}[2]{\ensuremath{F_{#2}\left(#1\right)}}
	\newcommand{\e}[1]{\ensuremath{e^{#1}}}
	\providecommand{\inv}[1]{\ensuremath{\frac{1}{#1}}}
	
	\bfr{Question}{
		In an exit poll of $900$ voters questioned, $360$ responded that they favor a particular
        proposition. On this basis, it was reported that $40\%$ of the voters favor the proposition.

        \begin{enumerate}[label = \brak{\textbf{\roman*}}]
            \item Find the margin of error if the confidence coefficient of the results is $0.95$.
            \item Find the confidence coefficient if the margin of error is $\mp 2\%$.
        \end{enumerate}
	}

    \bfr{Solution: \brak{\textrm{i}}}{
        Consider the random variables $x_i \forall i \in \sbrak{900}$, and the random variable $X = \Sigma_{i \in \sbrak{900}} x_i$.
        where $x_i$ is the boolean of a voter's opinion. Clearly, each $x_i$ is a Bernoulli variable
        with parameter $p$, and $X$ has a binomial distribution. Thus, the variance of $X$ is known, and equals $np\brak{1 - p}$, where n = $900$.

		The sample mean, $m$, is also evident from the question \brak{m = \frac{360}{900} = 0.4}.

        Therefore, the margin of error $\Delta$ is given by:
        \begin{align}
            \Delta &= \mp \sqrt{\frac{m\brak{1 - m}}{n}} z_\frac{\gamma + 1}{2} \label{form} \\
            &= \mp \frac{0.49}{30} \cdot 1.97 \nonumber \\
			&= \mp 3.2 \% \label{error}
        \end{align}
	}

	\bfr{Solution: \brak{\textrm{ii}}}{
		Given : $\Delta = \mp 2\%$. From ~\eqref{form}, 
		\begin{align}
			z_\frac{\gamma + 1}{2} &=  \sqrt{\frac{n}{m\brak{1 - m}}} \cdot \Delta \nonumber \\
				&= \frac{30 \cdot 0.02}{0.49} = 1.22 \nonumber \\
			\implies \frac{\gamma + 1}{2} &= \inv{\sqrt{2 \pi}}\int_{-\infty}^{1.22} \e{-\frac{z^2}{2}} dz = 0.89 \label{bigint}\\
			\implies \gamma = 0.78
		\end{align}

		The value of the integral in ~\eqref{bigint} was computed in Python.
	}

	\bfr{Graph}{
		\begin{figure}[h!]
			\caption{Cumulative function of Gaussian distribution}
			\includegraphics[scale = 0.5]{assig11.png}
		\end{figure}
	}
	\bfr{Result}{
		The results obtained:
		\begin{enumerate}[label = \brak{\textbf{\roman*}}]
            \item $\Delta = \mp 3.2 \%$
            \item $\gamma = 0.78$
        \end{enumerate}
	}

	
		
\end{document}
	
	

	
	
	
	