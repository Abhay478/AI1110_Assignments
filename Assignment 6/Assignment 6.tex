\let\negmedspace\undefined
\let\negthickspace\undefined
\documentclass[journal,12pt,twocolumn]{IEEEtran}
\usepackage{gensymb}
\usepackage{polynom}
\usepackage{amssymb}
\usepackage[cmex10]{amsmath}
\usepackage{amsthm}
\usepackage{stfloats}
\usepackage{bm}
\usepackage[misc]{ifsym}
%\setlength{\columnseprule}{1pt}
%\def\columnseprulecolor{\color{blue}}

\usepackage{longtable}
\usepackage{enumitem}
\usepackage{mathtools}
\usepackage[applemac]{inputenc}
\usepackage{tikz}
\usepackage[breaklinks=true]{hyperref}
\usepackage{listings}
    \usepackage{color}                                            %%
    \usepackage{array}                                            %%
    \usepackage{longtable}                                        %%
    \usepackage{calc}                                             %%
    \usepackage{multirow}                                         %%
    \usepackage{hhline}                                           %%
    \usepackage{ifthen}                                           %%
  
\usepackage{lscape}     
\usepackage{tfrupee}
\usepackage{parskip}

\pagecolor{white}
\DeclareMathOperator*{\Res}{Res}
\DeclareMathOperator*{\equals}{=}

\hyphenation{op-tical net-works semi-conduc-tor}
\def\inputGnumericTable{}  
\graphicspath{{./images/}}

\begin{document}

	\title{Assignment 6 : Example 12}
		\author{ Abhay Shankar K : cs21btech11001}

		\maketitle

		\bigskip

	\providecommand{\brak}[1]{\ensuremath{\left(#1\right)}}
	\providecommand{\sbrak}[1]{\ensuremath{\left[#1\right]}}
	\providecommand{\cbrak}[1]{\ensuremath{\left\{#1\right\}}}
	\newcommand{\solution}{\noindent \textbf{Solution: }}
	\newcommand{\question}{\noindent \textbf{Question: }}
	\providecommand{\pr}[2]{\ensuremath{P_{#1}\left(#2\right)}}
	
	
	\question
	
	Three coins are tossed simultaneously. Consider the events :
	
	\begin{itemize}
	
		\item E : Three heads or three tails
		
		\item F : At least two heads
		
		\item G : At most two heads
		
	\end{itemize}
	
	Which pairs of events are independent, and which are dependent?	
	
	
	\solution
	
	Let the random variable X represent the number of heads among the three tossed coins. 
	
	$\therefore X \in \cbrak{0, 1, 2, 3}$	
	
	Reframing the events in terms of X, 
	
	\begin{itemize}
	
		\item E : $X \in \cbrak{0, 3}$
		
		\item F : $X \in \cbrak{2, 3}$
		
		\item G : $X \in \cbrak{0, 1, 2}$
		
	\end{itemize}
	
	The probability mass function of X is as follows.
	
	\begin{figure}[h!]
	
		\caption{Probability mass function for X}
		
		\includegraphics[width = \columnwidth]{assig6_pmf}
		
			\label{pmf}
	
	\end{figure}
	
	The corresponding Cumulative distribution can be obtained for a given value of X by adding up the probabilities of all values of X less than the given value.
	
	Thus :
	
	\begin{figure}[h!]
	
		\caption{Cumulative Probability Function for X}
		
		\includegraphics[width = \columnwidth]{assig6_cdf}
		
			\label{cdf}
	
	\end{figure}

	
	From ~\ref{pmf}, we get
	\begin{align}
		\pr{}{E} &= \pr{X}{0} + \pr{X}{3} &= 0.25\\
		\pr{}{F} &= \pr{X}{2} + \pr{X}{3} &= 0.5\\
		\pr{}{G} &= \pr{X}{0} + \pr{X}{1} + \pr{X}{2} &= 0.875
	\end{align}
	
	Therefore, the probabilities of the events are :
	
	\begin{table}[h!]
	
		\caption{Probabilities}
		
		\include{tables/assig6_tab}
		
			\label{table}
	
	\end{table}
	
	Now, representing pairs of events with X,
	
	\begin{itemize}
	
		\item EF : $X = 3$
		
		\item FG : $X = 2$
		
		\item GE : $X = 0$
		
	\end{itemize}
	
	And their probabilities,
	\begin{align}
		\pr{}{EF} &= \pr{X}{3} &= 0.125 \\
		\pr{}{FG} &= \pr{X}{2} &= 0.375 \\
		\pr{}{GE} &= \pr{X}{0} &= 0.125
	\end{align}
	
	Which are tabularised below.
	
	\begin{table}[h!]
	
		\caption{Probabilities}
		
		\include{tables/assig6_tab2}
		
			\label{table2}
	
	\end{table}


	Checking for independence,
	
	\begin{align}
		\pr{}{E} \times \pr{}{F} &= 0.125 &= \pr{}{EF}\\
		\pr{}{F} \times \pr{}{G} &= 0.4375 &\neq \pr{}{FG}\\
		\pr{}{G} \times \pr{}{E} &= 0.21875 &\neq \pr{}{EG}
	\end{align}
	
	Hence, the events (E and F) are independent, whereas the events (F and G) and the events (G and E) are dependent.
				
\end{document}
	
	
	
	
	
	
	