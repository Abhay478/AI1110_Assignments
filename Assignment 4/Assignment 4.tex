\let\negmedspace\undefined
\let\negthickspace\undefined
\documentclass[journal,12pt,twocolumn]{IEEEtran}
\usepackage{gensymb}
\usepackage{polynom}
\usepackage{amssymb}
\usepackage[cmex10]{amsmath}
\usepackage{amsthm}
\usepackage{stfloats}
\usepackage{bm}
\usepackage[misc]{ifsym}

\usepackage{longtable}
\usepackage{enumitem}
\usepackage{mathtools}
\usepackage[applemac]{inputenc}
\usepackage{tikz}
\usepackage[breaklinks=true]{hyperref}
\usepackage{listings}
    \usepackage{color}                                            %%
    \usepackage{array}                                            %%
    \usepackage{longtable}                                        %%
    \usepackage{calc}                                             %%
    \usepackage{multirow}                                         %%
    \usepackage{hhline}                                           %%
    \usepackage{ifthen}                                           %%
  
\usepackage{lscape}     
\usepackage{tfrupee}
\usepackage{parskip}

\pagecolor{white}
\DeclareMathOperator*{\Res}{Res}
\DeclareMathOperator*{\equals}{=}

\hyphenation{op-tical net-works semi-conduc-tor}
\def\inputGnumericTable{}  
\graphicspath{{./images/}}

\begin{document}

	\title{Assignment 4 : Example 13}
		\author{ Abhay Shankar K : cs21btech11001}

		\maketitle

		\bigskip

	\providecommand{\brak}[1]{\ensuremath{\left(#1\right)}}
	\providecommand{\sbrak}[1]{\ensuremath{\left[#1\right]}}
	\providecommand{\cbrak}[1]{\ensuremath{\left\{#1\right\}}}
	\providecommand{\abs}[1]{\left\vert#1\right\vert}
	\providecommand{\norm}[1]{\left\lVert#1\right\rVert}
	\newcommand{\solution}{\noindent \textbf{Solution: }}
	\newcommand{\question}{\noindent \textbf{Question: }}
	\providecommand{\pr}[2]{\ensuremath{P_{#1}\left(#2\right)}}
	\newcommand*{\permcomb}[4][0mu]{{{}^{#3}\mkern#1#2_{#4}}}
	\newcommand*{\perm}[1][-3mu]{\permcomb[#1]{P}}
	\newcommand*{\comb}[1][-1mu]{\permcomb[#1]{C}}

	\newcommand{\myvec}[1]{\ensuremath{\begin{pmatrix}#1\end{pmatrix}}}
	\let\vec\mathbf


	\question
	
	A committee of two persons is selected from two men and two women. Find the probability that the committee will have :
	\begin{enumerate}[label = \brak{\textbf{\roman*}}]
		
		\item no men
		
		\item one man
		
		\item two men
		
	\end{enumerate}
	
	
	\solution
	
	Let the random variable X represent the number of men in the committee.
	
	\begin{table}[h!tb]
	
		\centering
		
		\caption{Range of X}
		\input{tables/assig4_rand}
		
		
		\label{table : range_table}
		
	\end{table}

	Let the set of possible candidates for the committee, i.e. the sample space, be denoted as S.
	
	\begin{align}
		S = \cbrak{m_1, m_2, w_1, w_2}
		 	\label{sample}
	\end{align}
	
	where \sbrak{m_i} are men and \sbrak{w_i} are women.
	
	
	Define the relation R on the set S to be :
	
	\begin{align}
		R = \cbrak{\brak{a, b} | \text{ a and b are both in the committee}}
			\label{relation}
	\end{align}
	
	R is also clearly the set of all subsets of S \brak{equation ~\eqref{sample}} of cardinality $2$.

	In roster form, we can represent the relation \brak{equation ~\eqref{relation}} as follows.
	
	
	\begin{align}
		R = \{\brak{w_1, w_2}, \brak{w_1, m_2}, \nonumber 
			\\
			\brak{m_1, w_2}, \brak{w_1, m_1}, \nonumber 
			\\
			\brak{m_2, w_2}, \brak{m_1, m_2}\}
				\label{roster}
	\end{align}
		
	Therefore, upon reviewing ~\eqref{roster}, the frequency distribution of the number of men in the committee \brak{X} is as given in table ~\ref{table : freq_table}.
	
	\begin{table}[h!t]
	
		\centering
		
		\caption{Frequency of X}
		\input{tables/dist_assig4}
		
		
		\label{table : freq_table}
		
	\end{table}
	
	
	The probabilities are then :
	
	\begin{enumerate}[label = \brak{\textbf{\roman*}}]

	\item $\pr{X}{0} = \frac{1}{6}$
	
	\item $\pr{X}{1} = \frac{4}{6} = \frac{2}{3}$
	
	\item $\pr{X}{2} = \frac{1}{6}$
	
	\end{enumerate}
	
	
	\textbf{Alternatively :}
	
	
	\begin{align}
		\pr{X}{r} = \frac{\brak{\comb{2}{r}}^2}{\comb{4}{2}}
			\label{comb_form}
	\end{align}
	
	Upon substitution of $r \in \cbrak{0, 1, 2}$ we get the same result as above.
	
	\begin{enumerate}[label = \brak{\textbf{\roman*}}]

		\item $\pr{X}{0} = \frac{1}{6}$
	
		\item $\pr{X}{1} = \frac{4}{6} = \frac{2}{3}$
	
		\item $\pr{X}{2} = \frac{1}{6}$
	
	\end{enumerate}

	
\end{document}
	
	
	
	
	
	
	