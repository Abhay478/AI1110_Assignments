\let\negmedspace\undefined
\let\negthickspace\undefined
\documentclass[journal,12pt,twocolumn]{IEEEtran}
\usepackage{gensymb}
\usepackage{polynom}
\usepackage{amssymb}
\usepackage[cmex10]{amsmath}
\usepackage{amsthm}
\usepackage{stfloats}
\usepackage{bm}
\usepackage[misc]{ifsym}
%\setlength{\columnseprule}{1pt}
%\def\columnseprulecolor{\color{blue}}

\usepackage{longtable}
\usepackage{enumitem}
\usepackage{mathtools}
\usepackage[applemac]{inputenc}
\usepackage{tikz}
\usepackage[breaklinks=true]{hyperref}
\usepackage{listings}
    \usepackage{color}                                            %%
    \usepackage{array}                                            %%
    \usepackage{longtable}                                        %%
    \usepackage{calc}                                             %%
    \usepackage{multirow}                                         %%
    \usepackage{hhline}                                           %%
    \usepackage{ifthen}                                           %%
  
\usepackage{lscape}     
\usepackage{tfrupee}
\usepackage{parskip}

\pagecolor{white}
\DeclareMathOperator*{\Res}{Res}
\DeclareMathOperator*{\equals}{=}

\hyphenation{op-tical net-works semi-conduc-tor}
\def\inputGnumericTable{}  
\graphicspath{{./images/}}

\begin{document}

	\title{Assignment 5 : Miscellaneous Exercise 10}
		\author{ Abhay Shankar K : cs21btech11001}

		\maketitle

		\bigskip

	\providecommand{\brak}[1]{\ensuremath{\left(#1\right)}}
	\providecommand{\sbrak}[1]{\ensuremath{\left[#1\right]}}
	\providecommand{\cbrak}[1]{\ensuremath{\left\{#1\right\}}}
	\providecommand{\abs}[1]{\left\vert#1\right\vert}
	\providecommand{\norm}[1]{\left\lVert#1\right\rVert}
	\newcommand{\solution}{\noindent \textbf{Solution: }}
	\newcommand{\question}{\noindent \textbf{Question: }}
	\providecommand{\pr}[2]{\ensuremath{P_{#1}\left(#2\right)}}
	\newcommand*{\permcomb}[4][0mu]{{{}^{#3}\mkern#1#2_{#4}}}
	\newcommand*{\perm}[1][-3mu]{\permcomb[#1]{P}}
	\newcommand*{\comb}[1][-1mu]{\permcomb[#1]{C}}

	\newcommand{\myvec}[1]{\ensuremath{\begin{pmatrix}#1\end{pmatrix}}}
	\let\vec\mathbf


	\question
	
	
	The number lock of a suitcase has $4$ wheels, each labelled with ten digits i.e. $0$ to $9$. The lock opens with a sequence of four digits with no repeats. What is the probability of a person getting the right sequence to open the suitcase?
	
	
	\solution
	
	
	Let $U = \cbrak{0, 1, 2, 3, 4, 5, 6, 7, 8, 9}$ be the sample space.
	
	
	Let the correct sequence $C = c_1 c_2 c_3 c_4$, and let the selected sequence $S = s_1 s_2 s_3 s_4$. 
	
	
	Let the random variables $X_1, X_2, X_3, X_4$ represent the boolean equality of the digits in the selected sequences, i.e. ,
	
	
	\begin{align}
		\forall i \in &\sbrak{4}, \nonumber \\
		c_i &= s_i \implies X_i = 1 \nonumber \\
		c_i &\neq s_i \implies X_i = 0 
			\label{rand}
	\end{align}
	
	
	$\therefore \textbf{Required value}  = \pr{X_1}{1} \times \pr{X_2}{1} \times \pr{X_3}{1} \times \pr{X_4}{1}$
	
	
	Progressing in ascending order,
	
	
	\begin{align}
		\pr{X_1}{1} = \frac{1}{\abs{U}} 
			\label{1sub} \\
		\pr{X_2}{1} = \frac{1}{\abs{U - \cbrak{c_1}}} 
			\label{2sub} \\
		\pr{X_3}{1} = \frac{1}{\abs{U - \cbrak{c_1, c_2}}} 
			\label{3sub} \\
		\pr{X_4}{1} = \frac{1}{\abs{U - \cbrak{c_1, c_2, c_3}}} 
			\label{4sub}
	\end{align}
	
	
	Substituting the values into equations ~\eqref{1sub}, ~\eqref{2sub}, ~\eqref{3sub} and ~\eqref{4sub}, 
	
	
	\begin{align}
		\pr{X_1}{1} = \frac{1}{10} 
			\label{1val} \\
		\pr{X_2}{1} = \frac{1}{9} 
			\label{2val} \\
		\pr{X_3}{1} = \frac{1}{8} 
			\label{3val} \\
		\pr{X_4}{1} = \frac{1}{7} 
			\label{4val}
	\end{align}
	
	
	Multiplying ~\eqref{1val}, ~\eqref{2val}, ~\eqref{3val} and ~\eqref{4val},
	
	
	\begin{align}
		\textbf{Required value}  &= \frac{1}{10 \cdot 9 \cdot 8 \cdot 7} \nonumber \\
						&= \underline{1.98 \cdot 10^{-4}} 
						\label{result}
	\end{align}
	
	
	\textbf{Alternatively,}
	
	
	We know that the number of sequences of $4$ digits formed from among $10$ digits without repetitions is
	
	
	\begin{align}
		N = \perm{10}{4} = 5040
			\label{formula}
	\end{align}
	
	
	So, the probability of the selected sequence matching the correct one is 
	
	
	\begin{align}
		\frac{1}{N} &= \frac{1}{\perm{10}{4}} \nonumber \\
				&= \underline{1.98 \cdot 10^{-4}}
				\label{alt_result}
	\end{align}
	
	
	Therefore, the probability of a person getting the right sequence to open the suitcase is $ \underline{1.98 \cdot 10^{-4}}$.
	
	
	\begin{table}[t]
	
		\centering
		\caption{Results of Python simulation}
		\input{tables/assig5_dist}
		
		\label{the_table}
		
	\end{table}
		
	
\end{document}
	
	
	
	
	
	
	